\documentclass{article}

\usepackage{setspace}
\onehalfspacing{}

\title{CMSC 124 Final Project}
\date{December 31, 2020}
\author{Tumulak, Patricia Lexa U., and Valles, Oscar Vian L.}

\setcounter{tocdepth}{2}

\begin{document}

\pagenumbering{gobble}
\maketitle
\newpage
\tableofcontents
\newpage
\pagenumbering{arabic}

\section{Javascript}
\subsection{Purpose and Motivations}
JavaScript or JS was first made to provide a lightweight programming language
for NetScape that would make web development more accessible instead of
requiring deeper training. Today, it is now one of the most widely used
programming languages and is mainly used to build websites and web-based
applications. This is because it allows the creation of interactive elements for
web pages that enhances the user experience. While HTML and CSS give web pages
structure, JavaScript gives it responsiveness that engages the user. It is also
not limited to just web technology but is also used in game development and
mobile applications.

\subsection{History}

  \subsubsection{Mocha}
  JavaScript was developed by Brendan Eich in September 1995 when he was tasked
  to develop a “Scheme for the web browser” --- a simple, dynamic, lightweight,
  and powerful scripting language with syntax that resembled Java for NetScape.
  It would be accessible to Snon-developers such as designers. The first version
  of JavaScript was made in only just 10 days. JavaScript was originally named
  Mocha, then called LiveScript, and finally renamed to JavaScript in December
  1995 to make it sound closer to Java and was presented as a scripting language
  for client-side tasks in the browser.

  \subsubsection{ES1 and 2}
  With Microsoft’s development of their own web browser, Internet Explorer, they
  developed their own language similar to Javascript called JScript. With the
  rapid growth of the internet, the need to standardize JavaScript was realized.
  NetScape tapped the European Computer Manufacturers Association (ECMA) to make
  a standardized language. In June 1997, the first version of the ECMAScript,
  labelled ECMA-262, was released. Due to trademark reasons, ECMA could not use
  JavaScript for the name of the standardized language and so, JavaScript is its
  commercial name.

  ECMAScript 2 or ES2 was released in June 1998 with relatively no new features
  to the language and only fixed a few inconsistencies between the ECMA and ISO
  standard for JavaScript.

  \subsubsection{ES3}
  ECMAScript 3 (or ES3) was released in December 1999 with changes to features
  were made such as regular expressions, exceptions and try/catch blocks,
  do-while block, the operators in and instanceof, and more.

  It was also during this time that AJAX (asynchronous JavaScript and XML) was
  born which was a technique that allowed pages to be updated asynchronously
  using JavaScript and browser built-in XMLHttpRequest object. The term AJAX was
  coined by Jesse James Garrett.

  \subsubsection{ES3.1 and ES4}
  As soon as ES3 was released in 1999, work on ES4 had already begun. The goal
  for this version of ECMAScript was to design features that allowed JavaScript
  to be used on the enterprise scale. However, conflict within the committee
  that worked on it (with representatives from Adobe, Mozilla, Opera, Microsoft,
  and Yahoo) started to arise. Some parties within expressed concern that ES4
  was beginning to get “too big and was out of control”. These were words by
  Douglas Crockford, an influential JavaScript developer from Yahoo. Microsoft
  also supported Doug’s concerns and eventually, the group split off to work on
  ES4 and another separate idea called ES3.1 which was a simpler proposal with
  no new syntax and only practical improvements. ES4 ended up being too complex
  and was finally scrapped in 2008. Eventually ES4 found its way into the market
  as ActionScript developed by Adobe which was the scripting language supported
  by Flash.

  jQuery is a JavaScript library that was initially released in August 2006.
  Created by John Resig, it allows developers to add extra functionality to
  webpages. According to W3Techs,
  74.4\% of the top 10 million websites use jQuery as of February 2020.

  NodeJS, a server-side runtime for JavaScript, was introduced in May 2009 by
  Ryan Dahl. This was built on Chrome’s V8 engine and it included an event loop.
  This helped build real-time web applications that scale. This also enabled
  developers to build a web app stack using only one programming language. This
  paradigm is called JavaScript Everywhere.

  \subsubsection{ES5}
  ECMAScript 3.1 was completed and released in December 2009, exactly 10 years
  after ES3. ECMAScript 3.1 was renamed ECMAScript 5 by the committee to avoid
  confusion. It was supported by Firefox 4, Chrome 19, Safari 6, Opera 12.10,
  and Internet Explorer 10. ES5 featured updates to the language such as
  getter/setters, reserved words, new methods for Object, Array, and Date, JSON
  support, among others. This did not require any changes to syntax.

  Another iteration of ES5 called ECMAScript 5.1 was released in 2011. However,
  this did not provide new features but only clarified ambiguous points.

  \subsubsection{ES6 (ES2015) and ES7 (ES2016)}
  2015 introduced a huge leap forward for JavaScript with the release of ES6 or
  ES2015 with the introduction of features such as promises, let and const
  bindings, generators, classes, arrow functions, spread syntax, among others.

  It was also during 2015 that ReactJS, the framework that solidified modern day
  declarative UI patterns, was introduced. It took some of the concepts of
  AngularJS with declarative UI but improved them unidirectional data flow,
  immutability, and the use of the virtual DOM.\@

  June 2016 saw the release of the 7th edition of ECMAScript --- ES2016. This was
  a smaller release with few new features introduced such as the exponential
  operator (**), keywords for asynchronous programming and the
  Array.prototype.includes function.

  \subsubsection{ES8, 9, 10 (ES2017, 2018, 2019)}
  For the next three years, more features were added in subsequent editions of
  ECMAScript. This included but not limited to features such as functions for
  easy Object manipulation, rest/spread operators for object literals,
  asynchronous iteration, and changes to Array.sort and Object.fromEntries.

  \subsubsection{ES11 (ES2020)}
  Published in June 2020, ECMAScript 2020 included new functions, the primitive
  type BigInt for integers that were arbitrarily sized, the nullish coalescing
  operator, and the globalThis object.

  \subsection{Language Features}

  \subsubsection{Java-like syntax}
  JS shares syntax with Java such as the use of brackets, semicolons to end
  statements, return statements, if and do..while statements

  \subsubsection{Dynamic typing}
  JS supports dynamic typing, allowing a variable’s type to be
  determined/defined based on the value stored

  \subsubsection{Prototypal Inheritance}
  JS uses prototypes instead of classes or inheritance. Unlike Java where we
  create a class then the objects for those classes, in JS, an object prototype
  is defined and then more objects can be made using the prototype.

  \subsubsection{Interpreted Language}
  JS script is interpreted by the JavaScript interpreter --- a built-in component
  of the web browser. In recent years however, just in time compilation is used
  for JS code such as in Chrome’s V8 engine.

  \subsubsection{Client-side validations}
  JS is a client-side scripting language. This means that JS functions can run
  even after the webpage has been loaded without communication with the server
  because the source code is processed by the client’s web browser instead of
  the web server. This makes JS very useful for things such as forms with the
  capability to validate errors in user input before sending the data to the
  server.

  \subsubsection{Let/Const}
  Unlike var which can be accessed outside of the function it was initialized
  in, let and const are blocked scope so they can only be accessed in the block
  they were defined in.

  \subsubsection{Arrow functions}
  Useful light-weight syntax that further simplified and shortened function
  syntax and lessened the number of lines of code

  \subsection{Paradigms}
  JavaScript is a multi-paradigm language. It supports both object-oriented
  programming as well as functional programming. With prototypal inheritance and
  object prototypes, this makes it object-oriented. The use of first-class
  functions, arrow functions (which are basically lambdas), and closures make
  the language functional (aka declarative) as well.

  \section{Rust}

  \subsection{Purpose}
  \subsection{History}
  \subsubsection{The Personal Years (2006--2010)}
  Graydon Hoare started working on a compiled, concurrent, safe, systems
  programming language as a hobby and as a research project. These were some of
  the following descriptions that Hoare wrote as Rust was beginning: Memory
  Safety, Typestate system, Mutability control, Side-effect control, and Garbage
  Control. Some of these remained throughout the years, some were also scrapped.
  At this point, \textasciitilde90\% of the language features and
  \textasciitilde70\% of the runtime was roughly working.

  \subsubsection{The Graydon Years (2010--2012)}
  During this time, Rust was adopted by Mozilla for use in their Servo project.
  This project was a rewrite of the Gecko rendering engine used for Firefox.
  Development of both Servo and Rust was parallel. Features were dogfooded by
  the Rust and Servo team and the language was iterated upon based on the
  feedback from both teams. During this time period Graydon became a benevolent
  dictator for life-like figure for Rust, much like how Linus Trovald is a
  benevolent dictator for life-like figure for Linux.

  \subsubsection{The Typesystem Years (2012--2014)}
  During this time period, the team grew and incorporated more experts that had
  experience with advanced type systems. Due to this, the type system also grew
  as well. As the type system grew, functionality found in the language was
  transferred into libraries. In addition, the package manager of the language,
  Cargo, was also implemented during this period. At this time, Graydon stepped
  down from the project. This allowed the project to be able to democratically
  improve the project, since a singular authority to determine what is best for
  the language is not present.

  \subsubsection{The Release Years (2015--May 2016)}
  Rust released 1.0.0-alpha back in January 9, 2015, 1.0.0-beta1 on February 16,
  2015 and a 1.0.0 on May 15, 2015. The 1.0.0 release guarantees that Rust as a
  language will continue to change, but it will do so in a backwards-compatible
  manner. This meant that stability will be maintained in its various versions.
  Four aspects of the language were also improved during this period: ecosystem,
  tooling, stability, and community.

  \subsubsection{The Production Years (May 2016--Present)}
  Nowadays, Rust has released version 1.48.0 and is widely used in various
  industries. Companies like Firefox, Dropbox, Cloudfare, and NPM all use Rust
  in some form or another. Rust has also gained popularity in developer circles,
  being the most loved programming language for two years in a row, according to
  StackOverflow. The number of Crates found in Crates.io has also ballooned to
  51,833, as of December 30, 2020.

  \subsection{Language Features}
  \subsection{Paradigms}





\end{document}
